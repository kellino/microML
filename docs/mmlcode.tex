A quasi-quicksort in microML\@. This does not sort in place, as is necessary for a true quicksort
implementation. Figure~\ref{fig:quicksort}

\begin{figure}[H]
    \begin{minted}[breaklines=true]{sml}
        let quicksort xs = 
            let smaller = \xs -> filter (\x -> x < head xs) (tail xs) in
            let greater = \xs -> filter (\x -> x >= head xs) (tail xs) in
            if xs == [] 
            then [] 
            else
            (quicksort (smaller xs)) ++ [head xs] ++ (quicksort (greater xs));
    \end{minted}
    \caption{Pseudo-Quicksort}
\label{fig:quicksort}
\end{figure}

Mergesort implemented in microML\@. Figure~\ref{fig:mergesort}

\begin{figure}[H]
    \begin{minted}[breaklines=true]{sml}
        let merge = \xs ys ->  
            if ys == [] then xs
            else if xs == [] then ys
            else if (head xs) <= (head ys) then (head xs):(merge (tail xs) ys)
            else (head ys):(merge xs (tail ys)) 

        let mergesort xs =
            let fsthalf = \xs -> take ((length xs) // 2) xs in 
            let sndhalf = \xs -> drop ((length xs) // 2) xs in
            if xs == [] then []
            else if (length xs) == 1 then xs
            else merge (mergesort (fsthalf xs)) (mergesort (sndhalf xs))
    \end{minted}
    \caption{Mergesort in microML}
\label{fig:mergesort}
\end{figure}

As far as possible, higher-order functions within the standard library are written with `foldr' as a
base. This provides a constant interface for all forms of list manipulation, and as such acts as the
higher-order `primitive' for microML\@. Figure~\ref{fig:foldr}

\begin{figure}[H]
    \begin{minted}[breaklines=true]{sml}
        let foldr f acc xs = if empty? xs then acc else ((f (head xs)) (foldr f acc (tail xs)))
        let map f xs = foldr (\x xs' -> (f x) : xs') [] xs
        let filter p xs = foldr (\x y -> if (p x) then (x:y) else y) [] xs
        let foldr1 f xs = foldr f (head xs) xs
    \end{minted}
    \caption{foldr in the standard library}
\label{fig:foldr}
\end{figure}


