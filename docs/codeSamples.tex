The following code extracts do not represent the complete code for microML. The interested reader is
referred to the repository on github \url{https://github.com/kellino/microML} or to the source code
included in the submission.

\subsubsection{Repl.hs}
The most complete entry point for the language, the repl environment, is partially listed here
\lstinputlisting[language=haskell, showstringspaces=false, basicstyle=\tiny, firstline=44]{../src/Repl/Repl.hs}

\subsubsection{CallGraph.hs}
\lstinputlisting[language=haskell, breaklines=true, showstringspaces=false, basicstyle=\tiny, firstline=20]{../src/Compiler/CallGraph.hs}

\subsubsection{Type checking of built-in functions}
Primitive operators and functions, such as head and tail are not available to the inference engine.
Therefore it was necessary to `hard code' the type signatures into the environment, thereby allowing
students to check the type signature of various important functions, but also meaning that these
functions can now be type checked\footnote{Haskell overcomes this problem partially by having almost
everything as a library function, including such things as + and -, which in microML are
primitives.}. Figure~\ref{fig:typesigs}

\begin{figure}[H]
    \begin{minted}[breaklines=true]{haskell}
  ("show"   , Forall  [polyA] (TVar polyA `TArrow` typeString))
, ("read"   , Forall  [] (typeString `TArrow` typeNum))
, ("ord"    , Forall  [] (typeChar `TArrow` typeNum))
, ("chr"    , Forall  [] (typeNum `TArrow` typeChar))
    \end{minted}
    \caption{Hard-coded type signatures, in Data.Map form}
\label{fig:typesigs}
\end{figure}

\subsubsection{Lexer.hs}
\lstinputlisting[language=haskell, breaklines=true, showstringspaces=false, firstline=13, basicstyle=\tiny]{../src/MicroML/Lexer.hs}

\subsubsection{Parser.hs}
An example of a parser written with parser combinators. Complex regular expressions have been replaced with small
regular expressions and other parser combinators.
\lstinputlisting[language=haskell, breaklines=true, showstringspaces=false, firstline=23, basicstyle=\tiny]{../src/MicroML/Parser.hs}

\subsubsection{Floating Point Representation}
Truncating floating-point representation in the repl is done using string manipulation. The choice
of three consecutive 0s is largely arbitrary. See Figure~\ref{fig:trunc} and
Section~\ref{floatingPoint}.
\begin{figure}
    \begin{minted}[breaklines=true]{haskell}
truncate' :: Double -> Double
truncate' = read . dropZeros . show
    where dropZeros x = head (split x) ++ "." ++ getValid (head (tail (split x)))
          split       = splitOn "."
          getValid s 
              | "e" `isInfixOf` s  = s
              | hasform s = if length s == 1 then s else  show $ read [head s] + 1
              | take 3 s   == "000" = "0"
              | otherwise  = head s : getValid (tail s) 

hasform :: String -> Bool
hasform (_:ys) = all (== '9') ys 
    \end{minted}
    \caption{The truncation function for doubles in the repl}
\label{fig:trunc}
\end{figure}

\subsubsection{Unit Testing\@: ListPrimitivesSpec.hs}
Unit testing was conducted with the HSpec package\footnote{\url{https://hspec.github.io/}}.
\lstinputlisting[language=haskell, breaklines=true, showstringspaces=false, basicstyle=\tiny]{../test/ListPrimitivesSpec.hs}

\subsubsection{Eval.hs}
\lstinputlisting[language=haskell, breaklines=true, showstringspaces=false, firstline=10,
basicstyle=\tiny]{../src/Repl/Eval.hs}

\subsubsection{Infer.hs}
\lstinputlisting[language=haskell, breaklines=true, showstringspaces=false, firstline=25, basicstyle=\tiny]{../src/MicroML/Typing/Infer.hs}


\section{Utilities}
MicroML also ships with a number of utility scripts.
The installation script is written in bash\footnote{The script makes use of bash arrays so it is not
100\% posix compatible. As \textit{dash} has now become the standard shell on Ubuntu future
iterations might need to take this into account.}.
\lstinputlisting[language=bash, breaklines=true, showstringspaces=false, basicstyle=\tiny]{../installMicroML}

MicroML also has a simple (neo)vim plugin which ships with the repo. The folding function is of some
interest, as it autofolds comments, setting the function name as its `title'. See
Figures~\ref{fig:fold} and~\ref{fig:foldVim}

\begin{figure}
    \begin{minted}[breaklines=true]{vim}
function! GetMMLFold(lnum) 
    let l:line = getline( a:lnum )
    " Beginning of comment
    if l:line =~? '\v^\s*--' || l:line =~? '\v^\s*(\*'
        return '2'
    endif
    if l:line =~? '\v^\s*$'
        let l:nextline = getline(a:lnum + 1)
        if l:nextline =~# '^--' || l:nextline =~? '^(\*'
            return '0'
        else
            return '-1'
        endif
    endif
    return '1'
endfunction 
    \end{minted}
    \caption{Part of the folding function for vim}
\label{fig:fold}
\end{figure}
